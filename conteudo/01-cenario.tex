%!TeX root=../tese.tex
%("dica" para o editor de texto: este arquivo é parte de um documento maior)
% para saber mais: https://tex.stackexchange.com/q/78101

\chapter{O Cenário do Estudo}

To-do

% O Capítulo 2, intitulado **"O Cenário do Estudo"**, tem o papel fundamental de contextualizar teoricamente o problema do seu TCC, que envolve **mobilidade urbana em São Paulo** e o **impacto de alagamentos** causados por chuvas extremas.

% A seguir, apresento um rascunho/estrutura detalhada do que você pode abordar em cada seção, seguindo o esqueleto fornecido:

% ***

% ## Capítulo 2: O Cenário do Estudo

% Este capítulo deve estabelecer a base teórica e o contexto geográfico e climático que definem o seu objeto de estudo (o subdistrito da Sé, a mobilidade de São Paulo e os alagamentos).

% ### 2.1 Panorama Geral da Mobilidade Urbana em São Paulo

% Esta seção introduz a complexidade do sistema de transporte da cidade.

% * **2.1.1 Introdução à Metrópole e à Mobilidade:**
%     * Apresente São Paulo como uma megacidade, destacando seus desafios de **infraestrutura e logística urbana**.
%     * Destaque a **matriz de transporte** predominante (transporte público vs. individual) e a dependência do modal rodoviário.
% * **2.1.2 Desafios Crônicos:**
%     * Aborde temas como **congestionamento** e o tempo médio de deslocamento.
%     * Mencione a **qualidade e a abrangência** dos transportes públicos (ônibus, metrô e trem).
% * **2.1.3 Importância da Resiliência:**
%     * Introduza a necessidade de um sistema de mobilidade **resiliente**, capaz de absorver e se recuperar de choques, como os causados por eventos climáticos extremos.

% ***

% ### 2.2 Conceito de Alagamento e a Relação com as Chuvas Extremas

% Esta seção deve ser o foco teórico do seu estudo sobre a causa do problema.

% * **2.2.1 Definição de Alagamento Urbano:**
%     * Defina o conceito de **"alagamento"** no contexto urbano, distinguindo-o de outros termos como inundação (se relevante para o seu referencial teórico).
%     * Descreva as **causas** de alagamento em São Paulo:
%         * Causas **hidrológicas** (impermeabilização do solo, características do relevo).
%         * Causas relacionadas à **infraestrutura** (capacidade de drenagem, falhas na manutenção da rede).
% * **2.2.2 Caracterização das Chuvas Extremas:**
%     * Conceitue **"chuvas extremas"** (eventos de alta intensidade e curta duração).
%     * Discuta a relação entre as **mudanças climáticas** e o aumento da frequência e intensidade desses eventos na região metropolitana.
% * **2.2.3 O Ciclo da Chuva e o Sistema de Drenagem:**
%     * Explique brevemente como o volume de chuva excede a **capacidade de escoamento** da rede de drenagem em pontos críticos da cidade, levando ao alagamento das vias.

% ***

% ### 2.3 Mudanças ao Longo do Tempo

% Esta seção deve trazer uma perspectiva histórica e evolutiva do problema.

% * **2.3.1 Histórico de Crescimento Urbano:**
%     * Apresente como a **expansão da cidade** e a consequente **impermeabilização** do solo ao longo do tempo potencializaram os problemas de drenagem.
% * **2.3.2 Evolução dos Registros de Alagamento:**
%     * Se possível, faça um panorama (mesmo que qualitativo, usando referências) da **evolução da ocorrência** de pontos de alagamento na cidade de São Paulo, mostrando o agravamento do cenário.
% * **2.3.3 Medidas Mitigatórias e Adaptação:**
%     * Descreva brevemente quais **medidas de planejamento** e obras (macro-drenagem, reservatórios, parques-piscinão) foram implementadas e como elas se relacionam com o cenário atual.

% ***

% ### 2.4 Impacto dos Alagamentos no Tráfego

% Esta seção conecta diretamente os dois principais temas do seu TCC.

% * **2.4.1 Interrupção Direta do Fluxo:**
%     * Descreva como o alagamento causa **interdição de vias** e a consequente redução da capacidade do sistema viário.
%     * Aborde a **diminuição da velocidade** e o risco de acidentes.
% * **2.4.2 Impacto na Rede e Efeitos de "Transbordo":**
%     * Explique como o bloqueio em um ponto causa **congestionamentos secundários** em vias alternativas (efeito cascata).
%     * Destaque o impacto nos **modais públicos**, como a interrupção de linhas de ônibus e, em casos extremos, de trechos de metrô/trem.
% * **2.4.3 Consequências Socioeconômicas:**
%     * Mencione brevemente as perdas de tempo (**atrasos**), os custos para os usuários e a cidade (combustível, poluição, prejuízo à economia). Esta parte deve ser um **preparo para a análise** que você fará no Capítulo 5.