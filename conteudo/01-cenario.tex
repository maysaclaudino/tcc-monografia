%!TeX root=../tese.tex
\chapter{O Cenário do Estudo}

De acordo com o Instituto Brasileiro de Geografia e Estatística (IBGE), a cidade de São Paulo alcançou 11,4 milhões de habitantes no censo de 2022, sendo classificada pela Organização das Nações Unidas (ONU) como uma \emph{megacidade} devido à sua expressiva população. Quando considerada a Região Metropolitana, esse número ultrapassa 22 milhões de habitantes, tornando-a a sexta área metropolitana mais populosa do mundo e a maior da América Latina.

Historicamente, o planejamento urbano paulistano privilegiou o transporte individual em detrimento do transporte coletivo. No início do século XX, a cidade substituiu os bondes elétricos por investimentos na expansão do sistema viário, promovendo o uso de automóveis particulares. O \emph{Plano de Avenidas}, formulado na década de 1930, consolidou esse paradigma ao reforçar o rodoviarismo e reduzir a prioridade dada ao transporte público. Atualmente, quase metade das viagens motorizadas realizadas no município são individuais.

O transporte público, por sua vez, enfrenta um processo contínuo de precarização, acompanhado de tarifas elevadas e de uma infraestrutura sobrecarregada. O tempo médio de deslocamento em transporte coletivo é, em média, 2,13 vezes superior ao registrado para o transporte individual. Na cidade de São Paulo, aproximadamente 5,5 milhões de passageiros são transportados diariamente por uma frota de pouco mais de 13 mil ônibus. A superlotação e a baixa qualidade do serviço contribuem para a migração de usuários para o transporte individual motorizado, seja automóvel ou motocicleta, na tentativa de reduzir o tempo de viagem e aumentar o conforto. Esse processo agrava os congestionamentos, intensificando a saturação do sistema viário urbano.

A construção de um sistema de mobilidade \emph{resiliente} é, portanto, essencial para garantir que os serviços de transporte continuem operando mesmo diante de choques externos, como eventos climáticos extremos. A resiliência da mobilidade urbana envolve a capacidade de resistir, absorver e recuperar-se de perturbações, assegurando a continuidade das atividades econômicas e sociais. Essa resiliência pode ser fortalecida por meio da integração de tecnologias de monitoramento de risco, sistemas de alerta e modelagem preditiva de cenários.

No contexto urbano, é fundamental diferenciar os conceitos de \emph{enchente}, \emph{inundação} e \emph{alagamento}. As enchentes são fenômenos naturais decorrentes do aumento da vazão de um curso d'água após uma precipitação intensa sobre sua bacia hidrográfica. As inundações ocorrem quando a água ultrapassa os limites naturais do leito do rio, atingindo áreas adjacentes normalmente não submersas. Já os alagamentos se caracterizam pelo acúmulo de água em vias, calçadas e outras infraestruturas urbanas, resultado da incapacidade do sistema de drenagem de escoar adequadamente o volume precipitado.

A cidade de São Paulo foi construída sobre e ao redor de seus principais rios Tietê, Pinheiros e Tamanduateí, cujas várzeas eram naturalmente sujeitas a inundações sazonais. A urbanização intensiva e a canalização dos cursos d'água alteraram profundamente o regime hidrológico da região, restringindo o espaço natural de escoamento das águas e agravando a frequência e intensidade dos alagamentos. A impermeabilização do solo urbano, aliada ao descarte inadequado de resíduos sólidos, reduz a infiltração e obstrui a microdrenagem, potencializando o acúmulo de água nas vias.

Áreas densamente urbanizadas, com alta concentração de superfícies impermeáveis e escassa vegetação, contribuem para a formação das chamadas \emph{tempestades de verão}, caracterizadas por chuvas intensas e de curta duração. Esses eventos extremos correspondem a precipitações significativamente acima da média histórica ou da referência climática local. Após o início dos anos 2000, observou-se um aumento na frequência e na intensidade dessas chuvas na Região Metropolitana de São Paulo.

As mudanças climáticas globais estão diretamente relacionadas ao aumento da ocorrência de chuvas extremas. O aquecimento global intensifica a capacidade da atmosfera de reter umidade, resultando em precipitações mais volumosas e concentradas em curtos períodos. Estima-se que, para cada aumento de 1°C na temperatura média global, a intensidade das precipitações extremas diárias se eleve em aproximadamente 7\%. Assim, projeta-se uma tendência de intensificação desses eventos ao longo das próximas décadas.

A infraestrutura de drenagem urbana de São Paulo é frequentemente insuficiente para suportar tais volumes. No centro expandido, as galerias pluviais são antigas, estreitas e deterioradas, incapazes de escoar o volume de chuva acumulado. Além disso, o acúmulo de detritos e o entupimento de bocas-de-lobo agravam o problema, provocando alagamentos mesmo em precipitações de baixa intensidade.

A combinação dos fatores urbanos e climáticos mencionados resultou na intensificação dos eventos de alagamento nas últimas décadas. Dados históricos indicam um aumento gradual do volume anual de chuvas em São Paulo: de cerca de 1.300 mm/ano no início do século XX para 1.500 mm/ano na década de 1990, com picos que ultrapassam 2.000 mm/ano em anos recentes.

Estudos conduzidos pelo Centro de Gerenciamento de Emergências (CGE) entre 2005 e 2019 registraram aproximadamente 15 mil pontos de alagamento no município. As ocorrências concentram-se principalmente na região central e nas proximidades das marginais dos rios Tietê e Pinheiros, com maior incidência nos meses de verão (dezembro a fevereiro).

Desde o século XIX, políticas de controle de cheias e de saneamento foram implementadas com o objetivo de liberar espaço para a expansão urbana e para a construção de infraestrutura de transporte. O \emph{Plano de Avenidas} de 1930 consolidou a ocupação dos fundos de vale por vias expressas, subordinando os cursos d'água à lógica do tráfego veicular. Atualmente, o município adota instrumentos como o \emph{Plano Diretor de Drenagem} (PDD) e o \emph{Plano Municipal de Redução de Riscos} (PMRR), que buscam mitigar os impactos das inundações por meio de obras estruturais, como reservatórios de detenção (\emph{piscinões}), e de soluções baseadas na natureza, como parques lineares. Entretanto, a fragmentação institucional e a persistência de paradigmas urbanos voltados ao transporte individual dificultam a adoção de uma gestão integrada e sustentável.

Os alagamentos urbanos têm como consequência imediata a interrupção parcial ou total do tráfego, ocasionando congestionamentos e comprometendo a mobilidade. Em muitos casos, vias importantes permanecem interditadas por horas, isolando bairros e comprometendo o acesso a serviços essenciais.

A redução da capacidade viária durante eventos de chuva decorre tanto de fatores físicos como o acúmulo de água nas pistas, quanto de alterações comportamentais dos motoristas, que tendem a reduzir a velocidade e aumentar a distância de segurança entre os veículos. Estudos indicam que, em condições chuvosas, a densidade na capacidade viária pode diminuir em cerca de 11\% em relação a dias de tempo normal.

Os bloqueios em vias estratégicas provocam efeitos em cascata, gerando congestionamentos secundários nas rotas alternativas. Quando as marginais dos rios Tietê e Pinheiros são afetadas, o tráfego regional é desviado para vias locais, que não possuem capacidade adequada, resultando em congestionamentos generalizados.

%%%%%%%%%%%%%%%%%%%%%%%%%%%%%%%%%%%%%%%%%%%%%%%%%%%%%%%%%%%%%%%%%%%%%%%%%%%%%%%%%%%%%%%%%%%%