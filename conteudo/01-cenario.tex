%!TeX root=../tese.tex
%("dica" para o editor de texto: este arquivo é parte de um documento maior)
% para saber mais: https://tex.stackexchange.com/q/78101

\chapter{O Cenário do Estudo}

To-do

% O Capítulo 2, intitulado **"O Cenário do Estudo"**, tem o papel fundamental de contextualizar teoricamente o problema do seu TCC, que envolve **mobilidade urbana em São Paulo** e o **impacto de alagamentos** causados por chuvas extremas.

% A seguir, apresento um rascunho/estrutura detalhada do que você pode abordar em cada seção, seguindo o esqueleto fornecido:

% ***

% ## Capítulo 2: O Cenário do Estudo

% Este capítulo deve estabelecer a base teórica e o contexto geográfico e climático que definem o seu objeto de estudo (o subdistrito da Sé, a mobilidade de São Paulo e os alagamentos).

% ### 2.1 Panorama Geral da Mobilidade Urbana em São Paulo

% Esta seção introduz a complexidade do sistema de transporte da cidade.

% * **2.1.1 Introdução à Metrópole e à Mobilidade:**
%     OK Apresente São Paulo como uma megacidade, 
%        destacando seus desafios de **infraestrutura e logística urbana**.
%        Destaque a **matriz de transporte** predominante (transporte público vs. individual) e 
%        a dependência do modal rodoviário.

o Instituto Brasileiro de Geografia e Estatistica (ibge) divulgou que a cidade de sao paulo atingiu 11,4 milhoes de habitantes no censo de 2022. a onu classifica a cidade como uma megacidade devido ao seu expressivo número de habitantes. considerando a regiao metropolitada, o numero sobe para 22 milhoes, colocando a regiao a sexta mais populosa do planeta e a primeira da america latina. historicamente o planejamento da cidade favoreceu o uso de transporte individual em detrimento do transporte coletivo. no passado, a cidade trocou os bondes para investir no sistema viario, incentivando o uso de automóveis.o Plano de Avenidas de 1930 também endossou o rodoviarismo e despriorizou o transporte coletivo. hoje quase metade das viagens motorizadas no municipio sao individuais.

% * **2.1.2 Desafios Crônicos:**
%     * Aborde temas como **congestionamento** e o tempo médio de deslocamento.
%     * Mencione a **qualidade e a abrangência** dos transportes públicos.

o serviço de transporte publico também tem sofrido precarizacao ao longo dos anos, alem das tarifas cada vez mais caras. O tempo médio de viagem em transporte coletivo é 2,13 vezes superior ao tempo médio em transporte individual. O transporte público absorve a maioria da demanda de passageiros na Região Metropolitana de São Paulo (RMSP). Na cidade de São Paulo, cerca de 5,5 milhões de passageiros diários são transportados por pouco mais de 13 mil ônibus. A superlotação e a baixa qualidade contribuem para o aumento do transporte individual motorizado, levando os usuários que podem pagar a optar por soluções individuais (automóvel ou motocicleta), buscando reduzir o tempo de viagem e aumentar o conforto e/ou a velocidade. os desafios que a infraestrutura mobilistica urbana da cidade apresenta resulta no aumento constante da frota de veiculos particulares, gerando consgestionamentos pela cidade. 

% * **2.1.3 Importância da Resiliência:**
%     * Introduza a necessidade de um sistema de mobilidade **resiliente**, capaz de absorver e se recuperar de choques, como os causados por eventos climáticos extremos.

A necessidade de um sistema de mobilidade resiliente é crucial para garantir que as cidades possam absorver, resistir e se recuperar de crises, desastres naturais e eventos extremos, mantendo o funcionamento contínuo dos serviços essenciais.
Um sistema resiliente deve ser capaz de lidar com choques externos, como inundações e alagamentos, que podem causar obstrução do tráfego horas. A resiliência pode ser fortalecida pela integração de tecnologias. oferecendo Monitoramento de Risco, alertas e previsao de eventos, Modelagem de Cenários

% ***

% ### 2.2 Conceito de Alagamento e a Relação com as Chuvas Extremas

% Esta seção deve ser o foco teórico do seu estudo sobre a causa do problema.

% * **2.2.1 Definição de Alagamento Urbano:**
%     * Defina o conceito de **"alagamento"** no contexto urbano, distinguindo-o de outros termos como inundação (se relevante para o seu referencial teórico).
%     * Descreva as **causas** de alagamento em São Paulo:
%         * Causas **hidrológicas** (impermeabilização do solo, características do relevo).
%         * Causas relacionadas à **infraestrutura** (capacidade de drenagem, falhas na manutenção da rede).

cabe aqui diferenciarmos os conceitos de enchente, inundacao e alagamento. enchentes sao um fenomeno natural da dinamica hidrológica dos rios, é o processo em que a vazão de um curso d'água aumenta após uma precipitação sobre a bacia hidrográfica. inundacoes decorrem da ocupação humana nessas áreas naturais de passagem de cheia acontecem quando um curso dagua submerge areas fora dos seus limites normais em zonas que normalmente não se encontram submersas. já os alagamentos acontecem quando se excede a capacidade de escoamento dos sistemas de drenagem e a agua se acumula na infraestrutura urbana como ruas e calçadas. 

a cidade foi construida sobre e em torno dos seus principais rios, Tiete, Pinheiros e Tamanduatei. naturalmente esses rios transbordavam inundando suas areas de varzea. com o avanço da ocupacao urbana ocupando as areas de varzea e a tentativa de domar o comportamento dos rios, as cheias que sempre ocorreram tornaram-se um problema. a urbanizacao da cidade trouxe consigo a canalização dos leitos fluviais que ignoram o percurso natural das águas agravando o problema das inundações e a intensa compactação e impermeabilização do solo que reduz a infiltração de água e aumenta o volume e a velocidade do escoamento superficial. além disso o descarte de residuos solidos e o lixo jogado nas ruas aumentam o risco de inundações e impedem a circulação da microdrenagem, provocando alagamentos.

% * **2.2.2 Caracterização das Chuvas Extremas:**
%     * Conceitue **"chuvas extremas"** (eventos de alta intensidade e curta duração).
%     * Discuta a relação entre as **mudanças climáticas** e o aumento da frequência e intensidade desses eventos na região metropolitana.

áreas com muito concreto e pouco verde, ajudam a formar as tempestades de verão, de curta duração e alta intensidade. Chuvas extremas são eventos de precipitação que se afastam da média daquilo que é considerado a condição normal ou da referência climática. depois dos anos 2000, esses eventos intensos estao acontecendo com uma frequencia maior.

as mudanças climaticas e o aumento da frequencia e intensidade das chuvas estao estreitamente relacionados. o aquecimento global é o principal elo entre eles pois aumenta a capacidade da atmosfera de reter umidade, fazendo com que a chuva quando cai, caia toda de uma vez aumentando a intensidade das chuvas. cada incremento adicional de 0,5°C no aquecimento global, aumenta perceptivelmente a intensidade e a frequência da precipitação intensa. Globalmente, a precipitação intensa diária é projetada para intensificar-se em cerca de 7\% para cada 1°C de aquecimento global. portanto a tendencia é que o incremento do aquecimento global faça com que os eventos de precipitação intensa se acentuem e se tornem mais frequentes na maioria das regiões

% * **2.2.3 O Ciclo da Chuva e o Sistema de Drenagem:**
%     * Explique brevemente como o volume de chuva excede a **capacidade de escoamento** da rede de drenagem em pontos críticos da cidade, levando ao alagamento das vias.

A infraestrutura de drenagem de águas pluviais é frequentemente precária ou insuficiente. as galerias pluviais do centro expandido da cidade sao antigos, estreitos e deteriorados, não dando vazão às águas, o que causa alagamentos mesmo com precipitações de baixa intensidade. Outro fator contribuinte é o acúmulo de detritos em galerias pluviais (canais de drenagem e cursos d’água). O lixo jogado nas ruas entope bueiros (bocas-de-lobo) e impede a circulação das águas, aumentando as cheias na microdrenagem.

% ***

% ### 2.3 Mudanças ao Longo do Tempo

% Esta seção deve trazer uma perspectiva histórica e evolutiva do problema.

% * **2.3.1 Histórico de Crescimento Urbano:**
%     * Apresente como a **expansão da cidade** e a consequente **impermeabilização** do solo ao longo do tempo potencializaram os problemas de drenagem.


% * **2.3.2 Evolução dos Registros de Alagamento:**
%     * Se possível, faça um panorama (mesmo que qualitativo, usando referências) da **evolução da ocorrência** de pontos de alagamento na cidade de São Paulo, mostrando o agravamento do cenário.

o conjunto dos fatores urbanos e climaticos apresentados contribuiram para a intensificacao de chuvas extremas e alagamentos nos últimos anos. Dados de longo prazo sinalizam uma clara tendência de aumento do volume anual de chuvas, com diminuição dos episódios de baixa intensidade e aumento das chuvas de forte intensidade. A média anual de chuvas aumentou, subindo de 1.300 mm/a no início do século para 1.500 mm/a na década de 1990, com picos chegando a 2.000 mm/a

Fulano de tal, analisou os Dados do Centro de Gerenciamento de Emergências entre 2005 e 2019 que registram 15 mil pontos de alagamento no municipio de sao paulo. A maior concentração desses alagamentos ocorre na região central e nas áreas sob as grandes vias marginais dos rios Pinheiros e Tietê. A maior ocorrência sazonal de alagamentos se dá no verão (dezembro, janeiro e fevereiro)

% * **2.3.3 Medidas Mitigatórias e Adaptação:**
%     * Descreva brevemente quais **medidas de planejamento** e obras (macro-drenagem, reservatórios, parques-piscinão) foram implementadas e como elas se relacionam com o cenário atual.

Desde o século XIX, medidas de saneamento e controle de cheias transformaram os rios para liberar espaço para a expansão urbana e o desenvolvimento de infraestruturas de energia e transporte. Projetos como o Plano de Avenidas (1930) consolidaram as avenidas de fundo de vale, subordinando os cursos d'água à circulação de veículos e levando ao aumento na frequência e intensidade das inundações ao longo do tempo. Atualmente, o planejamento busca enfrentar esses problemas através de instrumentos como o Plano Diretor de Drenagem (PDD) e o Plano Municipal de Redução de Riscos (PMRR), que orientam a execução de obras, incluindo tanto soluções estruturais tradicionais (como reservatórios de detenção, ou piscinões) quanto a incorporação de Soluções Baseadas na Natureza (SbN) e Parques Lineares para maior resiliência. Contudo, a persistência de paradigmas antigos e a fragmentação da gestão, que frequentemente privilegiam o transporte individual e os interesses imobiliários em detrimento de uma agenda socioambiental integrada, dificultam a efetivação e a continuidade dessas ações de longo prazo, mantendo as populações mais vulneráveis expostas a riscos hidrológicos

% ***

% ### 2.4 Impacto dos Alagamentos no Tráfego

% Esta seção conecta diretamente os dois principais temas do seu TCC.

% * **2.4.1 Interrupção Direta do Fluxo:**
%     * Descreva como o alagamento causa **interdição de vias** e a consequente redução da capacidade do sistema viário.
%     * Aborde a **diminuição da velocidade** e o risco de acidentes.

A consequência mais direta dos alagamentos e inundações é a interrupção e paralisação do tráfego. pode haver horas de interrupção do tráfego, frequentemente isolando zonas ou bairros. As condições climáticas adversas, como a chuva, afetam diretamente o comportamento do tráfego e a capacidade operacional das vias

A redução da capacidade do sistema viário em dias chuvosos ocorre devido a uma combinação de fatores físicos e a alterações significativas no comportamento dos motoristas, que impactam diretamente a velocidade e a densidade do fluxo de tráfego.
Em condições chuvosas, os motoristas tendem a Diminuir suas velocidades e Aumentar a distância de segurança entre os veículos.
O aumento da distância de segurança resulta na diminuição da densidade de veículos que a via pode suportar na capacidade
Em estudos realizados em rodovias paulistas, a chuva provocou uma redução média de 11\% na densidade na capacidade em comparação com o tempo bom. a densidade na capacidade é o número de veículos que ocupam uma unidade de comprimento de uma faixa ou via em um determinado instante.

% * **2.4.2 Impacto na Rede e Efeitos de "Transbordo":**
%     * Explique como o bloqueio em um ponto causa **congestionamentos secundários** em vias alternativas (efeito cascata).
%     * Destaque o impacto nos **modais públicos**, como a interrupção de linhas de ônibus e, em casos extremos, de trechos de metrô/trem.

Quando uma via é interrompida, a interrupcao dissemina rapidamente seu impacto em congestionamentos por diversas outras vias de circulação da cidade desencadeando um efeito cascata de congestionamentos secundários, devido ao redistribuimento da demanda de tráfego. Se vias importantes (como as marginais) são interrompidas, o tráfego de longa distância ou regional é forçado a usar vias urbanas alternativas que não possuem capacidade para esse volume, resultando em congestionamentos generalizados

% * **2.4.3 Consequências Socioeconômicas:**
%     * Mencione brevemente as perdas de tempo (**atrasos**), os custos para os usuários e a cidade (combustível, poluição, prejuízo à economia). Esta parte deve ser um **preparo para a análise** que você fará no Capítulo 5.

%%%%%%%%%%%%%%%%%%%%%%%%%%%%%%%%%%%%%%%%%%%%%%%%%%%%%%%%%%%%%%%%%%%%%%%%%%%%%%%%%%%%%%%%%%%%