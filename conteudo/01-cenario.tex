%!TeX root=../tese.tex
\chapter{O Cenário do Estudo}

De acordo com o Instituto Brasileiro de Geografia e Estatística (IBGE), a cidade de São Paulo alcançou 11,4 milhões de habitantes no censo de 2022, sendo classificada pela Organização das Nações Unidas (ONU) como uma \emph{megacidade} devido à sua expressiva população \hl{\verbatim{\cite{ibge2022,onu2018}}}. Quando considerada a Região Metropolitana, esse número ultrapassa 22 milhões de habitantes, tornando-a a sexta área metropolitana mais populosa do mundo e a maior da América Latina.

Historicamente, o planejamento urbano paulistano privilegiou o transporte individual em detrimento do transporte coletivo. Segundo \citeauthor{marcosnona}, a substituição dos bondes elétricos por investimentos na expansão do sistema viário, iniciada no início do século XX, marcou uma mudança de paradigma na mobilidade urbana paulistana. O \emph{Plano de Avenidas}, elaborado por \hl{\verbatim{Prestes Maia}} na década de 1930, consolidou essa orientação ao priorizar o rodoviarismo e expandir o sistema viário em detrimento da rede de transporte público \hl{\verbatim{\cite{maia1930,rolnik2001}}}. Atualmente, quase metade das viagens motorizadas realizadas no município são individuais \hl{\verbatim{\cite{metro2023}}}.

O transporte público, por sua vez, vem enfrentando um processo contínuo de precarização. De acordo com \citeauthor{zioni_gomes_moraes_2024}, o tempo médio de deslocamento por transporte coletivo é 2,13 vezes superior ao registrado para o transporte individual. Estima-se que cerca de 5,5 milhões de passageiros sejam transportados diariamente por uma frota de pouco mais de 13 mil ônibus \cite{zioni_gomes_moraes_2024}. A superlotação, a irregularidade nos horários e o custo elevado das tarifas têm incentivado a migração para o transporte individual motorizado, o que agrava os congestionamentos e intensifica a saturação do sistema viário urbano.

A construção de um sistema de mobilidade resiliente é essencial para assegurar a continuidade das atividades econômicas e sociais mesmo diante de eventos extremos. Conforme apontam \cite{silva2024resiliencia}, a resiliência da mobilidade urbana envolve a capacidade de resistir, absorver e se recuperar de perturbações, garantindo o funcionamento mínimo dos sistemas de transporte. Essa resiliência pode ser fortalecida por meio da integração de tecnologias de monitoramento, sistemas de alerta e modelagem preditiva de cenários \hl{\verbatim{\cite{zhang2015resilience}}}.

No contexto urbano, é importante diferenciar enchentes, inundações e alagamentos. \hl{\verbatim{Segundo Tucci (2007)}}, as enchentes resultam do aumento natural da vazão de cursos d'água após chuvas intensas, as inundações ocorrem quando os rios transbordam e atingem áreas não usualmente submersas, e os alagamentos decorrem da incapacidade da drenagem urbana de escoar o volume precipitado, acumulando água em vias e calçadas \hl{\verbatim{\cite{tucci2007}}}. 

São Paulo foi construída sobre e ao redor dos rios Tietê, Pinheiros e Tamanduateí, cujas várzeas eram naturalmente sujeitas a inundações sazonais. A canalização desses cursos d'água e a urbanização intensa modificaram profundamente o regime hidrológico da região, restringindo o espaço natural de escoamento das águas e aumentando a frequência dos alagamentos. A impermeabilização do solo e o descarte inadequado de resíduos sólidos reduzem a infiltração e obstruem o sistema de drenagem, potencializando o acúmulo de água nas vias.

Além disso, a expansão de áreas densamente urbanizadas favorece a formação das chamadas tempestades de verão, caracterizadas por chuvas intensas e de curta duração. \hl{\verbatim{Segundo Marengo et al. (2020)}}, esses eventos correspondem a precipitações acima da média histórica e têm se tornado mais frequentes na Região Metropolitana de São Paulo \hl{\verbatim{\cite{marengo2020}}}. 

As mudanças climáticas globais estão diretamente relacionadas ao aumento da ocorrência de chuvas extremas. O aquecimento global intensifica a capacidade da atmosfera de reter umidade, resultando em precipitações mais volumosas e concentradas em curtos períodos. O \citeauthor{ipcc2008mudancas} aponta que, para cada aumento de 1°C na temperatura média global, a intensidade das precipitações extremas diárias pode aumentar em até 7\%. Assim, projeta-se uma tendência de intensificação desses eventos ao longo das próximas décadas.

A infraestrutura de drenagem paulistana  é insuficiente para o volume atual de precipitações. No centro expandido, as galerias pluviais são antigas, estreitas e, muitas vezes, deterioradas \cite{jacobi_2000}. O acúmulo de resíduos sólidos e o entupimento de bocas de lobo reduzem ainda mais a capacidade de escoamento, resultando em alagamentos mesmo em chuvas de baixa intensidade.

Dados históricos \hl{\verbatim{do Centro de Gerenciamento de Emergências (CGE, 2020)}} mostram um aumento gradual do volume anual de chuvas na cidade, passando de 1.300 mm/ano no início do século XX para cerca de 1.500 mm/ano na década de 1990, com picos superiores a 2.000 mm/ano em anos mais recentes. Entre 2005 e 2019, o CGE registrou aproximadamente 15 mil pontos de alagamento no município, com concentração nas marginais Tietê e Pinheiros e nos na região central \cite{carvalho2018analise}. 

Desde o século XIX, políticas públicas voltadas ao controle de cheias e saneamento foram implementadas para viabilizar a expansão urbana. Segundo  \hl{\verbatim{Somekh (2013)}}, o \emph{Plano de Avenidas} transformou os fundos de vale em eixos viários, subordinando os cursos d'água à lógica do tráfego veicular. Mais recentemente, planos como o Plano Diretor de Drenagem e o Plano Municipal de Redução de Riscos buscaram mitigar os impactos das inundações com obras estruturais, como piscinões, e com soluções baseadas na natureza \cite{prefeituradesãopaulo2025}. No entanto, a fragmentação institucional e a predominância do paradigma rodoviarista ainda dificultam uma gestão integrada e sustentável das águas urbanas.

Os alagamentos urbanos têm como consequência imediata a interrupção parcial ou total do tráfego, provocando congestionamentos e comprometendo a mobilidade. Vias importantes da capital podem permanecer interditadas por horas, afetando o acesso a serviços e a atividades econômicas. A redução da capacidade viária decorre tanto do acúmulo físico de água nas pistas quanto do comportamento dos motoristas, que tendem a reduzir a velocidade e aumentar a distância de segurança \cite{cardoso2017relacao}. 

Segundo \citeauthor{cardoso2017relacao}, a densidade da capacidade viária pode diminuir em até 11\% durante eventos de chuva intensa. Os bloqueios em vias estratégicas, como as marginais Tietê e Pinheiros, produzem efeitos em cascata, sobrecarregando as rotas alternativas e gerando congestionamentos generalizados.
